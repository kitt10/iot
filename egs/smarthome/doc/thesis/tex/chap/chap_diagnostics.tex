\chapter{Diagnostika a detekce anomálií} \label{chap:methods}

Proč řešit diagnostiku čidel v rámci chytré domácnosti, ... \\
Úvod do detekce anomálií, základní principy, smysl využití, ... \\
Diagnostika a detekce anomálií v projektu chytré domácnosti probíhá na 3 úrovních ... \\

\section{Detekce chyb na úrovni ESP8266} \label{sec:example_xor}

První a nejnižší úroveň diagnostiky jednotlivých senzorů \\
Implementace detekce chyb na úrovni samotného microchipu \\
Popis základního principu - při načítání ze dat ze senzoru čidlo pozná, když je senzor nefunkční - nevrací  žádné hodnoty a ESP pošle info o nefunkčním senzoru) \\
Robustnost - odolnost ESP vůči výpadkům senzorů - esp nespadne kvůli nefunkčnímu senzoru, jen změní status zprávy, ... \\
Nonstop provoz - senzory lze k ESP připojovat a odpojovat v reálném čase bez nutnosti vypínání nebo restartu microchipu, pokud senzor v jakémkoliv časovém okamžiku odpojím - změní se status zprávy na "error", ve chvíli kdy senzor zase připojím na desku - esp začne měřit a posílat, ... \\

\section{Detekce anomálií na základě klasifikace} \label{sec:example_xor}

Implementace funkcí scikit-learn pro natrénovaní modelů \\
Nastavení modelu a natrénování modelů pro jednotlivé veličiny \\
Porovnání schémat natrénovaných modelů \\
Implementace do projektu - klasifikace jednotlivých příchozích zpráv (přidání atributu "classification" do každé zprávy)
Implementace detekce anomálií přijatých zpráv na základě rozhodnutí klasifikace - klasifikace 1 nebo -1 (1 v případě, že přijatá zpráva svou hodnotou "odpovídá" natrénovanému modelu, -1 v případě, že se vychyluje od natrénovaného modelu) \\
Přenos této informace o klasifikaci do webového rozhraní \\

\section{Diagnostika stavu čidel na serveru} \label{sec:example_xor}

Implementace detekce anomálií na úrovni serveru (brokeru) \\
Implementace funkce sensor-check() \\
Periodická kontrola odesílání zpráv jednotlivých čidel - pokud čidlo z neznámého důvodu neodešle zprávu nebo pokud se zpráva nepřenese k brokeru - informace o anomálii se přenese do webového rozhraní \\
Vysvětlení stavů čidel - 3 možné stavy čidla - ok, value-error, error \\

Závěr diagnostiky: kontrola stavu jednotlivých čidel + kontrola věrohodnosti posílaných hodnot + kontrola pravidelnosti odesílaných zpráv