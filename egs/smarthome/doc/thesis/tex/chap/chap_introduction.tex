\chapter{Úvod} \label{chap:introduction}
Využití moderních technologií v domácnosti otevírá možnosti monitorování a ovládání domácího prostředí a jeho okolí dříve nevídanými způsoby. Chytrá domácnost reaguje skrze systém automatického řízení na potřeby uživatele a pomocí efektivního nakládání s energiemi snižuje provozní náklady. Mezi vlastnosti inteligentního domu patří především zajištění komfortu pro jeho obyvatele a ekonomická efektivita. Centrem chytré domácnosti je komplexní systém, který sbírá lokálně měřená data a na základě těchto dat automatizuje provoz v domácnosti. Tento řídící systém se stará například o centrální řízení teploty v domě, zabezpečení objektu nebo o automatické ovládání osvětlení. Hlavní využití této centrálně řízené domácnosti spočívá v autonomii - inteligentní dům rozhoduje na základě uživatelského nastavení a dat a ke svému fungování nepotřebuje aktivní zásahy od uživatele. Tento projekt se zaměřuje na jeden z aspektů inteligentního domu - na konstrukci chytrých senzorů a následné zpracování naměřených dat. \par  
Díky chytrým senzorům, které monitorují fyzikální veličiny v domácnosti a jejím okolí, je možné vzdáleně kontrolovat jednotlivé sekce v inteligentním domě, optimalizovat provozní náklady a zvýšit zabezpečení objektu. Naměřené hodnoty lze ukládat, sledovat jejich vývoj v čase a na základě klasifikace predikovat budoucí vývoj a detekovat anomálie. \par
Cílem této práce je sestrojit funkční model chytré domácnosti, který svým zaměřením a funkčností odpovídá realitě a má konkrétní reálné aplikace. Tento model se skládá z několika senzorů měřících různé fyzikální veličiny a řídícího systému. Významnou část toho projektu tvoří systém automatické diagnostiky komunikace, který využívá principů strojového učení a na základě klasifikace poskytuje další informace o senzorech a měřených veličinách. Celý projekt je uživateli zpřístupněný přes webové rozhraní, které přehledně poskytuje všechny dostupné informace. 

\section*{Cíle práce} \label{sec:thesis_objectives}
Cíle této práce jsou následující. 

\begin{enumerate}
   \item Otestovat funkcionalitu senzorů vhodných pro využití v projektu chytré domácnosti.
   \item Zajistit fyzické zapojení vybraných senzorů a jejich komunikaci s mikročipem ESP8266.
   \item Pomocí protokolu MQTT zprovoznit komunikaci vybraných čidel s webovým rozhraním pro vzdálené monitorování.
   \item Navrhnout systém automatické diagnostiky MQTT zpráv s cílem detekce výpadků senzorů a jiných anomálií.
\end{enumerate}

\section*{Současný stav řešené problematiky} \label{sec:state_of_the_art}
\cite{scikit-learn} Jednou ze stěžejních částí této práce je systém automatické diagnostiky, jehož primárním úkolem je detekovat chyby a anomálie. Detekce anomálií je proces, jehož účelem je identifikovat neočekávané prvky v množině dat. Tyto neočekávané prvky jsou charakterizovány výchylkami od normy ostatních dat. Pro umožnění detekce anomálií musejí být naplněny dva předpoklady: 

\begin{enumerate}
	\item Neočekávané události (hodnoty) se v datech vyskytují výjimečně.
	\item Neočekávané hodnoty se od standardních dat odlišují významně.
\end{enumerate}

Tím, že se neočekávané hodnoty od ostatních dat podstatně odlišují a vyskytují se jen výjimečně, je možné tyto anomálie spolehlivě detekovat. V problematice detekce anomálií se rozlišují dvě základní situace: 

\begin{itemize}
	\item V množině trénovacích dat jsou přítomny anomálie, které jsou definovány jako prvky ležící daleko od ostatních dat; Algoritmus pro detekci výchylek ignoruje tyto přítomné anomálie a snaží se zaměřit na oblasti, kde je nejvíce dat (tato situace se nazývá \textit{Outlier Detection}).
	\item Trénovací data nejsou znečištěna anomáliemi a algoritmus rozhoduje o tom, zda nové prvky korespondují s trénovací množinou nebo zda se jedná o anomálie (tato klasifikace je nazývána \textit{Novelty Detection}).
\end{itemize}

Detekce anomálií má v reálném světě četné využití skrze obory jako je bankovnictví, medicína nebo automatická výroba. V objemných databázích je často velmi složité objevit opakující se vzory a identifikovat anomálie, proto se uchylujeme k použití algoritmů strojového učení. V současné době je k dispozici řada algoritmů, které slouží k detekci anomálií. Několik příkladů \cite{scikit-learn} je uvedeno níže.  \par

\begin{itemize}
	\item \textit{Robust Covariance} je algoritmus založený na předpokladu, že sledovaná data mají známé rozložení (například Gaussovské) a že je možné odhadnout tvar tohoto rozložení; Na základě elipsy, která prochází centrálními daty jsou za anomálie označeny prvky, které leží za hranicí této elipsy. 
	\item Algoritmus \textit{One-Class SVM} vyžaduje výběr středu z množiny dat a zvolení parametrů pro definování hranice; Za anomálie jsou označeny prvky, které se nacházejí za hranicí.
	\item \textit{Local Outlier Factor} je algoritmus, který na základě úrovně abnormálnosti (velikost odchylky) vypočítává skóre, které přiřazuje jednotlivým prvkům; Skóre je vypočítáváno v závislosti na počtu sousedních prvků v okolí. 
	\item V tomto projektu je pro detekci anomálií využitý algoritmus \textit{Isolation Forest}, jehož historie sahá do roku 2008. 
\end{itemize}

\subsection*{Algoritmus Isolation Forest} \label{subsec:isolation_forest}
Isolation Forest \cite{scikit-learn} je algoritmus založený na strojovém učení, který detekuje anomálie tím, že identifikuje a izoluje neočekávané hodnoty od ostatních dat. Tento algoritmus nepotřebuje předem definovanou množinu dat, která je brána jako bezchybná a vzorová. Ostatní algoritmy vyžadují trénovací množinu dat, ve které se nevyskytují žádné anomálie a za anomálie považují data, která jsou mimo předem učenou normovanou množinu dat. Isolation Forest detekuje anomálie bez předem definované množiny normálních dat na základě dvou výše zmíněných předpokladů. Mezi klíčové vlastnosti algoritmu Isolation Forest patří: 

\begin{enumerate}
  \item Nízké nároky na výpočetní paměť a rychlost detekce
  \item Schopnost zpracovávat mnohodimenzionální data bez další informace o typu dat (algoritmus nepotřebuje vědět jaký typ dat zpracovává).
  \item V trénovacích datech mohou nebo nemusejí být přítomny anomálie.
\end{enumerate}

Princip detekce anomálií algoritmu Isolation Forest spočívá v přiřazení číselného skóre všem datům v množině dat. Na množině trénovacích dat je natrénován model, který následně klasifikuje nově příchozí vzorky - na základě číselného skóre provádí rozhodnutí, zda daný prvek koresponduje s natrénovaným modelem nebo zda se jedná o anomálii. Klasifikace (rozhodnutí) probíhá podle následující rozhodovací funkce:

\[
    f(x)= 
\begin{cases}
     \ \ 1 \ (\text{sample \textit{ok}}) & \text{if } S > 0 \\
     \ \ 0 \ (\text{sample \textit{outlier}}) & \text{if } S \leq 0 \\
\end{cases}
\]

kde $x$ je daný vzorek (sample) a $S$ je hodnota číselného skóre. Po přiřazení skóre vzorkům v množině dat jsou za anomálie označena data, jejichž hodnoty přesáhnou předem určený práh (práh je zde nula). Pokud je skóre vzorku kladné, vzorek koresponduje s natrénovaným modelem a je v pořádku - nejedná se o anomálii. Pokud je skóre záporné, daný vzorek je považován za tzv. \textit{outlier} - neodpovídá natrénovanému modelu a je označen za anomálii. \par
Schopnost algoritmu zpracovávat a vyhodnocovat data bez předchozí znalosti obsahu dat je klíčová pro klasifikaci různých druhů veličin. V tomto projektu jsou pro algoritmus Isolation Forest data formována dvěma způsoby: 

\begin{itemize}
	\item jednodimenzionálně: [timestamp] - obsah dat je tvořen jedním atributem, který je klasifikován; Zde je klasifikován čas odeslání zprávy. 
	\item dvoudimenzionálně: [timestamp, value] - obsah dat je tvořen dvěma atributy; Zde je klasifikována hodnota měřené veličiny v daném časovém okamžiku.
\end{itemize}

Klasifikace jedno- a dvoudimenzionálních dat je popsána v kapitole \cref{chap:diagnostics}. Algoritmus Isolation Forest je vhodný pro klasifikaci obou typů veličin a jeho hlavní výhodou jsou nízké nároky na výpočetní výkon a rychlost detekce anomálií.  Anomálií je v tomto projektu:

\begin{itemize}
	\item Událost, která vznikla v čase, kdy je vznik této události nepravděpodobný. 
	\item Hodnota měřené veličiny, která se v daném časovém okamžiku výrazně odlišuje od ostatních hodnot v množině dat.
\end{itemize}

\section*{Obsah práce} \label{sec:thesis_outline}
Práce se skládá z šesti kapitol a svou strukturou odpovídá standardu vědeckých publikací. \par
V první kapitole jsou popsány aktuální trendy v oblasti chytré domácnosti, základní principy detekce anomálií a stručně vystiženy cíle tohoto projektu. \par
V \cref{chap:hardware} je popsán veškerý použitý hardware a na schématech je zobrazena konstrukce jednotlivých senzorů v projektu chytré domácnosti. Tato část dále popisuje programování síťové komunikace mikročipu \textit{ESP8266} se serverem a využití počítače \textit{Raspberry Pi}. \par 
V \cref{chap:network_database} jsou znázorněny datové toky v celém projektu a popsán komunikační protokol \textit{MQTT}. Tato kapitola je věnována vysvětlení principu přenosu dat po síti a jejich ukládání do databáze. \par
Systém automatické diagnostiky a detekce anomálií je popsán v \cref{chap:diagnostics}. Tato kapitola je rozdělena do tří částí podle úrovně diagnostického systému. Je zde popsána detekce chyb na úrovni mikročipu ESP8266, systém kontroly periodicity příchozích zpráv a použití klasifikátoru pro detekci anomálií v měřených datech. \par
Vizualizace veškerých naměřených dat a stavu senzorů je ukázána v \cref{chap:web_page}. \par
V příloze \ref{app:diagnostics} jsou uvedeny ukázky grafů, které jsou vygenerovány na základě reálných dat pomocí natrénovaných modelů pro jednotlivé veličiny. \par
V příloze \ref{app:structure_of_the_repository} je zobrazena adresářová struktura repozitáře v projektu chytré domácnosti. 
