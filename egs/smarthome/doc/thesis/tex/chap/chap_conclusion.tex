\chapter{Závěr} \label{chap:conclusion}
V projektu senzorického řešení chytré domácnosti s automatickou diagnostikou komunikace byl vytvořen funkční model chytré domácnosti skládající se z pěti senzorů, které měří celkem osm fyzikálních veličin uvnitř domu a ve venkovním prostředí. Součástí modelu je systém automatické diagnostiky stavu čidel a detekce anomálií. I přes množství problémů při hardwarové konstrukci senzorů a následně při softwarovém vývoji se podařilo projekt dokončit v plánovaném rozsahu a vytvořit funkční simulaci chytré domácnosti.  \par
V první části projektu šlo o návrh hardwarového řešení a fyzickou realizaci jednotlivých senzorů. V dalším kroku bylo potřeba naprogramovat komunikační logiku pro všechny senzory, určit hierarchii jednotlivých komponent v této práci, implementovat \textit{MQTT} protokol a realizovat ukládání dat do databáze. Následně byla řešena problematika automatické diagnostiky komunikace na třech úrovních - detekce chyb na úrovni samotného mikročipu \textit{ESP8266}, kontrola periodicity příchozích zpráv a detekce anomálií na základě klasifikace. Na závěr projektu byla vytvořena komplexní webová vizualizace. Hlavním výstupem této práce je přehledné zobrazení aktuálních hodnot pozorovaných veličin doplněné o informace o stavu jednotlivých senzorů a věrohodnosti naměřených dat. \par
V budoucnu je možné na tento projekt navázat a dále rozšiřovat základnu chytrých senzorů a automatizační logiku. Nad aplikací diagnostiky komunikace jednotlivých senzorů se serverem by dále šlo naprogramovat automatizační logiku, která by skrze aktivní prvky ovládala části domácnosti. Tato aplikační logika by mohla být například ve formě neuronové sítě, která by se v průběhu času učila návyky uživatele a automatizovala domácnost. \par
Kromě pasivních prvků, kterými jsou senzory měřící okolní veličiny, je možné chytrou domácnost doplnit o aktuátory, které na základě informací ze senzorů budou konat akce. Aktuátorem v chytré domácnosti by mohl být například mikročip ovládající vnitřní osvětlení nebo žaluzie. Tyto aktivní prvky by využívaly informace z již sestrojených senzorů a celý projekt by se stal komplexnější. \par 
V projektu senzorického řešení chytré domácnosti byl kladen důraz na univerzálnost řešení s možností navázání další práce v budoucnu, ať už z pohledu hardwarového rozšíření o další senzory nebo rozvinutí aplikační logiky. Cíl tohoto projektu, tj. automatické shromažďování dat v domácnosti za účelem jejich grafické vizualizace bez nutnosti uživatelské obsluhy, se podařilo naplnit. Celý systém aktuálně běží několik měsíců plně autonomně bez výpadků.