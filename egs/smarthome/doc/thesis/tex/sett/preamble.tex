%%%%%%%%%%%%%%%%%%%%%%%%%%%%%%%%%%%%%%%%%
% Masters/Doctoral Thesis 
% LaTeX Template
% Version 2.3 (25/3/16)
%
% This template has been downloaded from:
% http://www.LaTeXTemplates.com
%
% Version 2.x major modifications by:
% Vel (vel@latextemplates.com)
%
% This template is based on a template by:
% Steve Gunn (http://users.ecs.soton.ac.uk/srg/softwaretools/document/templates/)
% Sunil Patel (http://www.sunilpatel.co.uk/thesis-template/)
%
% Template license:
% CC BY-NC-SA 3.0 (http://creativecommons.org/licenses/by-nc-sa/3.0/)
%
%%%%%%%%%%%%%%%%%%%%%%%%%%%%%%%%%%%%%%%%%

%----------------------------------------------------------------------------------------
%	PACKAGES AND OTHER DOCUMENT CONFIGURATIONS
%----------------------------------------------------------------------------------------

\documentclass[
11pt, 						% The default document font size, options: 10pt, 11pt, 12pt
oneside, 					% Two side (alternating margins) for binding by default, uncomment to switch to one side
%chapterinoneline,			% Have the chapter title next to the number in one single line
czech, 					% ngerman for German
singlespacing, 				% Single line spacing, alternatives: onehalfspacing or doublespacing
%draft, 					% Uncomment to enable draft mode (no pictures, no links, overfull hboxes indicated)
nolistspacing, 				% Uncomment this to set spacing in lists to single
%liststotoc, 				% Uncomment to add the list of figures/tables/etc to the table of contents
%toctotoc, 					% Uncomment to add the main table of contents to the table of contents
parskip, 					% Uncomment to add space between paragraphs
%nohyperref, 				% Uncomment to not load the hyperref package
headsepline, 				% Uncomment to get a line under the header
hidelinks,
]{style/thesis_style} 		% The class file specifying the document structure

\usepackage[utf8]{inputenc} % Required for inputting international characters
\usepackage[T1]{fontenc} 	% Output font encoding for international characters

\usepackage{lmodern} 		% Font family

% Use the bibtex backend with the authoryear citation style (which resembles APA) (citestyle=alphabetic)
\usepackage[backend=bibtex,style=numeric,citestyle=authoryear-comp,natbib=true,sorting=ynt]{biblatex} 
\addbibresource{bib/thesis_bibliography.bib} 				% The filename of the bibliography

% Required to generate language-dependent quotes in the bibliography
\usepackage[autostyle=true]{csquotes}

%----------------------------------------------------------------------------------------
%	MARGIN SETTINGS
%----------------------------------------------------------------------------------------

\geometry{
	paper=a4paper, 			% Change to letterpaper for US letter
	inner=2.5cm, 			% Inner margin
	outer=3.8cm, 			% Outer margin
	bindingoffset=2cm, 		% Binding offset
	top=1.5cm, 				% Top margin
	bottom=1.5cm, 			% Bottom margin
	%showframe,				% show how the type block is set on the page
}

% --------------------------------------
% Kitt added style and packages
% --------------------------------------
\usepackage[table,xcdraw]{xcolor}
\definecolor{lightgray}{gray}{0.95}
\definecolor{lightblue}{rgb}{0.6,0.75,0.85}
\definecolor{lightyellow}{rgb}{0.95,1.0,0.8}

\usepackage{courier}

\usepackage{listings}
\lstset{frame=single, basicstyle=\footnotesize\ttfamily, backgroundcolor=\color{lightgray}}
\renewcommand{\lstlistingname}{Part of Code}
\renewcommand{\lstlistlistingname}{List of Algorithms and Code Parts}

\usepackage{amsmath}
\usepackage{cleveref}
\crefname{listing}{code part}{code parts}
\crefname{figure}{Obr.}{Obr.}
\crefname{table}{Tab.}{Tab.}
\crefname{equation}{Eq.}{Eq.}
\crefname{section}{Kap.}{Kap.}


\usepackage{float}

\usepackage{epstopdf}

\usepackage{commath}

\usepackage{multirow}

\usepackage{url}

\newcommand{\pluseq}{\mathrel{+}=}

\usepackage{caption}
\usepackage{subcaption}

\usepackage{tocloft}
\addtolength{\cftchapnumwidth}{5pt}

\makeatletter
\def\l@figure{\@dottedtocline{1}{1.5em}{3em}}
\makeatother

\usepackage{pdfpages}

\usepackage[most]{tcolorbox}

\usepackage{enumitem}

\usepackage{forest}

\usepackage{longtable}

\usepackage{array} % for defining a new column type
\usepackage{varwidth} %for the varwidth minipage environment
\newcolumntype{M}{>{\begin{varwidth}{4cm}}l<{\end{varwidth}}}

\def\xstrut{\rule[-2ex]{0pt}{5ex}}
