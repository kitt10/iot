\chapter{Automatické ovládání žaluzií} \label{chap:regresory}
    Pro automatické ovládání žaluzií byly navrženy tři regresory, které odhadují vhodné nastavení žaluzie (výška vytažení (position) a míra naklopení (tilt)) na základě 15 místného příznakového vektoru (případně časové posloupnosti těchto vektorů). Použité regresory byly nazvány \uv{If-else}, \uv{FFNN} a \uv{LSTM} podle svého principu. Tato kapitola se zabývá jejich návrhem, specifickými vlastnostmi, které z něj vyplývají, a jeho metodami.
    
    Poslední dva zmíněné regresory (FFNN a LSTM jsou neuronové sítě (\acrshort{nn}). K práci s jejich modely bylo využíváno rozharní Keras v jazyce Python, které umožňuje jejich učení, vyhodnocování i následné použití pro regresi (\cite{keras:doc}). Při volbě jejich vnitřní struktury a parametrů učení se slepě zkoušely natrénovat sítě se všemi možnými kombinacemi určitých parametrů a následně se vybrala nejlepší z nich ve smyslu minimální střední kvadratické odchylky (\acrshort{mse}) na testovací datové sadě (\cref{ssec:test}). Parametry a prohledávané hodnoty jsou uvedeny v \refskl{sec:ffnn,sec:lstm}{sekcích}.

    Všechny regresory mají jednotné rozhraní a jimi doporučené řízení poskytují prostřednictvím metod \code{control()} a \code{predict()}. První z nich je určena pro jednotlivé vzorky, druhá pak pro jejich seznam. Obě se volají při obsluhování příslušných požadavků z uživatelského rozhraní při použití Simulátoru (\cref{sec:sim}) nebo grafu řízení na záložce Live (\cref{sec:live}). Metoda \code{predict()} se od metody \code{control()} liší tím, že u \acrshort{nn} se celá vstupní data zpracovávají najednou, což má pozitivní vliv na rychlost. % Kvůli architektuře použité knihovny Keras je ale nutné mít správně nastavený počet přijímaných vzorků ve vstupní vrstvě modelu, proto se při každém požadavku odvozuje nový model. Ten má stejné vnitřní parametry jako původní a poskytuje tak pro daná data stejné výsledky.
    
    Přetrénování regresorů založených na \acrshort{nn} je možné spustit pomocí metody \code{train()} na předaných datech.

    \section{Příprava dat pro strojové učení} \label{sec:data_prep}
        Naměřená data jsou uložená v databázi (\cref{chap:dataCollection}) ve svých původních jednotkách a jednotlivě podle okamžiku pořízení. Pro jejich využití při strojovém učení byla data načtená z databáze přeškálována a vytvořily se z nich vzorky dle architektury jednotlivých \acrshort{nn}. Vzorky byly náhodně a disjunktně rozděleny do 3 datových sad (trénovací, validační a testovací) v poměru 8:1:1. Výše popsané činnosti vykonával skript \file{mongo2h5.py}, který na závěr 3 datové sady pro každou síť uložil do souboru pro pozdější využití.
        
        Hodnoty jednotlivých veličin byly převedeny do intervalu $\left\langle 0,1\right\rangle$ posunutím a škálováním očekávaného intervalu jejich hodnot uvedeného v \refskl{tab:features}{tabulce} v \refskl{chap:features}{kapitole}. Pro regresor FFNN každý vzorek v datových sadách odpovídal jednomu naměřenému vzorku, zatímco pro regresor LSTM každý obsahoval příznaky ze 64 časových okamžiků a požadované hodnoty nastavení žaluzie pro poslední z nich. Uložený soubor s datovými sadami byl ve formátu h5 a jeho název obsahoval informaci o časovém intervalu, ze kterého data v něm pochází, a o architektuře \acrshort{nn}, pro kterou je určen.

        Datová sada se disjunktně dělí na 3 menší datové sady. Každá z nich slouží k různým účelům při vývoji systému.
        \paragraph{Trénovací datovou sadu}tvoří 80\% náhodně vybraných vzorků a při trénování jsou její vzorky po dávkách předkládány \acrshort{nn}. V každém kroku je pro ně vyhodnocována ztrátová funkce, jejíž hodnoty slouží k nastavení vnitřních parametrů učicím algoritmem.
        \paragraph{Validační datovou sadu}tvoří 10\% náhodně vybraných vzorků. V každém kroku učení se na ní vyhodnocuje několik metrik. Neslouží přímo pro učení \acrshort{nn} - ověřuje se na ní, zda síť dobře zobecňuje a nedochází k přetrénování na trénovací datovou sadu (overfitting). Vybíral se vždy doposud nejlepší model podle hodnoty ztrátové funkce vyhodocené na validační datové sadě.
        \paragraph{Testovací datová sada}sestává z 10\% náhodně vybraných vzorků. Byla použita pro vyhodnocení úspěšnosti učení sítí s různou strukturou při hledání té nejvhodnější (\cref{sec:ffnn,sec:lstm}) a také pro celkové vyhodnocení regresorů a vlivu příznaků na odhad regresorů (\cref{sec:res_regr,sec:res_retr,sec:res_features}).
            \label{ssec:test}

    \section{Regresor založený na pravidlech (If-else)}
        Základní regresor \uv{If-else} tvoří výstup deterministicky na základě pravidel vyjádřených ve formě zřetězených konstrukcí if-elif-else (odtud tedy název) v programovacím jazyku Python. Slouží k porovnání navržených metod strojového učení s rozhodováním dle neměnných pravidel, které se v současnosti v praxi v domácí automatizaci obvykle používá (\cite{apple:home}; \cite{openhab:openhab}; \cite{hass:hass}).

        Pravidla byla konstruována podle slovy formulovaných požadavků uživatele na stav žaluzií v závislosti na aktuálních podmínkách (vyjádřených příznakovým vektorem) s ohledem na to, že konkrétní parametry může být nutné změnit, protože je uživatel není schopen přesně definovat, nastavují se proto v konfiguračním souboru \file{cfg\_ifelse.yml}. Formulována byla tato pravidla pro výšku vytažení žaluzie:
        \begin{enumerate}
            \item Ve dne\footnote{Den znamená, že je vnější intenzita osvětlení vyšší než nastavená mez a zároveň je čas vyšší než nastavený čas vstávání v daný den týdne. Jinak je noc.}:
            \begin{enumerate}
                \item Během jara a podzimu\footnote{Jaro a podzim jsou dny v roce z intervalů $\left\langle 80, 134\right\rangle \cup  \left\langle 274, 320\right\rangle $}: Je-li chladno\footnote{Chladno znamená, že je teplota venkovního vzduchu nižší než nastavená mez. Zvolena byla podle zkušenosti s vytápěním použitého domu na 10$^\circ$C.}, žaluzie má být vytažená, jinak v případě střední teploty\footnote{Střední teplota znamená že je teplota venkovního vzduchu v intervalu mezi zmíněnou mezí pro \uv{chladno} a další nastavitelnou mezí, která byla zvolena na 18$^\circ$C.} se má rozhodovat na základě rozdílu mezi teplotami předpovězenými za 2 a 1~hodinu (růst teploty). Pokud je růst teploty v nastaveném pásmu nebo vyšší nebo je teplota vnějšího vzduchu vyšší než obě meze, žaluzie má být ve stejné pozici jako by bylo léto (následující bod), v případě růstu nižšího se má vytáhnout.\footnote{Nižší hranice pásma byla zvolena na $-1{,}5~^\circ \mathrm{C}$, vyšší pak na $1{,}5~^\circ \mathrm{C}$.}
                \item V létě\footnote{Léto jsou dny v roce z intervalu $\left\langle 135, 273\right\rangle$}: Pokud to dovolují podmínky a uživatel je doma, žaluzie má být vytažená, jinak zatažená. Podmínky vhodné pro vytažení žaluzie jsou takové, kdy nehrozí přehřátí interiéru - rozdíl azimutu slunce a azimutu směru kolmého na žaluzii (směrem od domu) v absoultní hodnotě je vyšší než 90$^\circ$ nebo je předpovězená nejvyšší denní teplota nižší, než nastavená mez (například z důvodu oblačnosti, deště atp.).
                \item V zimě\footnote{Zima jsou ostatní dny, tedy dny v roce z intervalů $\left\langle 321, 366\right\rangle \cup \left\langle 1, 79\right\rangle $}: Žaluzie má být vytažená.
            \end{enumerate}
            \item V noci v kterémkoli období má být žaluzie zatažená.
        \end{enumerate}
        Kromě pravidel pro výšku vytažení byla formulována také pravidla pro míru naklopení:
        \begin{enumerate}
            \item Přes den:
            \begin{enumerate}
                \item Během jara a podzimu: V případě potřeby (vysoká intenzita osvětlení venku a zároveň alespoň vysoká teplota vzduchu uvnitř nebo teplota venkovního vzduchu, vše vzhledem k nastavitelným mezím) se má žaluzie naklopit jako by bylo léto, jinak mají být její lamely vodorovně.
                \item Během léta se má žaluzie naklopit o nejmenší možný úhel vzhledem k vodorovné rovině tak, aby zabránila (je-li to vzhledem k vzájemné poloze Slunce a žaluzie nutné) průchodu přímého slunečního záření oknem. Jestliže je ale intenzita osvětlení nižší než nastavená mez (vlivem počasí, nebo času), mají být lamely žaluzie vodorovně.
                \item V zimě je žaluzie vytažená a úhel naklopení nemá smysl uvažovat, z konstrukce bude žaluzie vodorovně.
                \item V případě, že uživatel není doma, naklopení žaluzie (jiné než vodorovné) má odpovídat $\frac{8}{10}$ jinak definovaného naklopení.
            \end{enumerate}
            \item V noci: Žaluzie úplně naklopená, vyjímkou je případ, kdy je vnitřní teplota nižší než nastavená mez - to může znamenat, že se uživatel snaží větrat a je tak vhodné žaluzie otevřít jen tak, aby mezi lamelami mohl lépe proudit vzduch, ale i přes to bránily průchodu světla pod malými úhly (po ulici projíždějící auta atp.).
        \end{enumerate}

        Z návrhu vyplývá, že se tento regresor nepřizpůsobuje novým návykům uživatele, protože využívá v čase neměnných pravidel. Některá pravidla mají volitelné parametry, které lze ladit a dosáhnout tak přesněji uživatelem požadovaných výsledků. 
    \section{Dopředná neuronová síť jako regresor (FFNN)} \label{sec:ffnn}
        První regresor založený na neuronových sítích byl nazvaný \uv{FFNN} podle anglického sousloví \emph{Feedforward Neural Network}, které označuje dopřednou neuronovou síť. Při volbě její vnitřní struktury byly prohledávány všechny kombinace hodnot parametrů uvedených v \refskl{tab:ffnnhyper}{tabulce}.
        % Table generated by Excel2LaTeX from sheet 'FFNN'
\begin{table}[htbp]
  \centering
    \begin{tabular}{l|l}
    \toprule
    Parametr & Seznam hodnot \\
    \midrule
    \midrule
    struktura sítě & [10, 15], [15, 10], [10, 15, 10], [20, 15, 10], [10, 15, 20] \\
    \midrule
    počet epoch & 500, 200, 700 \\
    \midrule
    velikost dávky & 16, 32 \\
    \midrule
    ztrátová funkce & MSE, kosinová podobnost \\
    \bottomrule
    \bottomrule
    \end{tabular}%
    \caption[Prohledáváné parametry dopředné neuronové sítě]{Prohledáváné parametry dopředné neuronové sítě. Tabulka zachycuje 4 parametry neuronové sítě a jejího učení jejichž vzájemné kombinace se prohledávaly za účelem získání nejlepší z nich pro úlohu regrese hodnot nastavení venkovní žaluzie na základě 15 příznaků. Hodnoty parametru struktura sítě jsou seznamy (v hranatých závorkách) počtu neuronů v jednotlivých vrstvách sítě. \acrshort{mse} značí střední kvadratickou odchylku.}
  \label{tab:ffnnhyper}%
\end{table}%
        Všechny vrstvy sítě jsou plně propojené. Neurony v nich využívají aktivační funkci ReLU, kromě výstupní vrstvy, která je aktivována funkcí sigmoid (obě funkce jsou popsané v \refskl{ssec:activationFcn}{sekci}), která je omezená shora i zdola a výstup vynásobený koeficientem je tak možné přímo použít jako případný akční zásah.
    \section{Rekurentní neuronová síť jako regresor (LSTM)} \label{sec:lstm}
        Tento regresor je rekurentní neuronová síť, která používá \emph{Long Short-Term Memory} (\acrshort{lstm}) buňky (\cref{ssec:rnn}). Vnitřní struktura byla volena na základě výsledků prohledávání parametrů (popsáno v úvodu \hyperref[chap:regresory]{této kapitoly}), prohledávané hodnoty každého z nich jsou uvedené v \refskl{tab:lstmhyper}{tabulce}.
        % Table generated by Excel2LaTeX from sheet 'LSTM'
\begin{table}[htbp]
  \centering
    \begin{tabular}{l|l}
    \toprule
    Parametr & Seznam hodnot \\
    \midrule
    \midrule
    struktura sítě & [64, 32], [64, 16], [64, 64], [64, 16, 16] \\
    \midrule
    počet epoch & 100, 200 \\
    \midrule
    velikost dávky & 16, 64 \\
    \midrule
    ztrátová funkce & MSE, kosinová podobnost \\
    \bottomrule
    \bottomrule
    \end{tabular}%
    \caption[Prohledáváné parametry rekurentní neuronové sítě]{Prohledáváné parametry rekurentní neuronové sítě. Tabulka zachycuje 4 parametry neuronové sítě a jejího učení jejichž vzájemné kombinace se prohledávaly za účelem získání nejlepší z nich pro úlohu regrese hodnot nastavení venkovní žaluzie na základě 15 příznaků. Každá z hodnot parametru struktura sítě je seznam (ohraničený hranatými závorkami) o délce rovné počtu vrstev sítě z nichž každá má takový počet neuronů jako odpovídající prvek seznamu. \acrshort{mse} značí střední kvadratickou odchylku.}
  \label{tab:lstmhyper}%
\end{table}%

        Aktivační funkcí je v tomto případě ve všech vrstvách hyperbolický tangens a funkce sigmoid pro rekurentní krok. Výstupní vrstva je stejná jako v případě \acrshort{nn} regresoru FFNN.
    \section{Přetrénování neuronových sítí (retraining)} \label{sec:retraining}
        Přetrénováním se rozumí opakované trénování \acrshort{nn} regresorů na všech v daný okamžik dostupných datech. Přetrénování probíhá na již existujícím modelu, který se jen dále upravuje a zpřesňuje. Automaticky se v backendu plánuje na půlnoc každého dne, ale je ho možné také ručně spustit v záhlaví každé ze stránek \acrshort{gui}. Díky tomu se systém může přizpůsobovat novým okolnostem v datech, kterými mohou být zejména nové návyky uživatele.

        Aby bylo přetrénování rychlé, používá se méně epoch a větší batch size než při běžném trénování. Obě \acrshort{nn} se přetrénovávají s batch size 128, \acrshort{ffnn} 50 epoch, \acrshort{lstm} pak 2 epochy.