\chapter{Sběr dat a komunikace komponent} \label{chap:dataCollection}
  Použitým algoritmům strojového učení je nutné dodat data sestávající z~posloupnosti hodnot příznaků (více v \refskl{chap:features}{kapitole}) a odpovídajících stavů žaluzií (výška vytažení a sklon lamel) v~daných časových okamžicích. Tato kapitola se zabývá metodami sběru dat v~jednotlivých součástech systému, přenosem z~nich a jejich vzájemnou komunikací.
  
  \begin{figure}[H]
    \centering
    % Graphic for TeX using PGF
% Title: C:\Users\vojta\Documents\Škola\bakalarka\iot\egs\smartblinds\doc\img\comm.dia
% Creator: Dia v0.97.2
% CreationDate: Wed Apr 06 21:03:56 2022
% For: vojta
% \usepackage{tikz}
% The following commands are not supported in PSTricks at present
% We define them conditionally, so when they are implemented,
% this pgf file will use them.
\ifx\du\undefined
  \newlength{\du}
\fi
\setlength{\du}{15\unitlength}
\begin{tikzpicture}
\pgftransformxscale{1.000000}
\pgftransformyscale{-1.000000}
\definecolor{dialinecolor}{rgb}{0.000000, 0.000000, 0.000000}
\pgfsetstrokecolor{dialinecolor}
\definecolor{dialinecolor}{rgb}{1.000000, 1.000000, 1.000000}
\pgfsetfillcolor{dialinecolor}
\pgfsetlinewidth{0.050000\du}
\pgfsetdash{}{0pt}
\pgfsetdash{}{0pt}
\pgfsetmiterjoin
\definecolor{dialinecolor}{rgb}{0.000000, 0.000000, 0.000000}
\pgfsetstrokecolor{dialinecolor}
\draw (16.460703\du,7.976279\du)--(16.460703\du,8.976279\du)--(21.590703\du,8.976279\du)--(21.590703\du,7.976279\du)--cycle;
\pgfsetlinewidth{0.050000\du}
\pgfsetdash{}{0pt}
\pgfsetdash{}{0pt}
\pgfsetmiterjoin
\definecolor{dialinecolor}{rgb}{0.000000, 0.000000, 0.000000}
\pgfsetstrokecolor{dialinecolor}
\draw (10.195703\du,7.976279\du)--(10.195703\du,8.976279\du)--(13.635703\du,8.976279\du)--(13.635703\du,7.976279\du)--cycle;
\pgfsetlinewidth{0.050000\du}
\pgfsetdash{}{0pt}
\pgfsetdash{}{0pt}
\pgfsetmiterjoin
\definecolor{dialinecolor}{rgb}{0.000000, 0.000000, 0.000000}
\pgfsetstrokecolor{dialinecolor}
\draw (2.895703\du,8.426279\du)--(2.895703\du,10.151279\du)--(7.280703\du,10.151279\du)--(7.280703\du,8.426279\du)--cycle;
\pgfsetlinewidth{0.050000\du}
\pgfsetdash{}{0pt}
\pgfsetdash{}{0pt}
\pgfsetmiterjoin
\definecolor{dialinecolor}{rgb}{0.000000, 0.000000, 0.000000}
\pgfsetstrokecolor{dialinecolor}
\draw (8.115703\du,4.826279\du)--(8.115703\du,5.826279\du)--(14.095703\du,5.826279\du)--(14.095703\du,4.826279\du)--cycle;
\pgfsetlinewidth{0.050000\du}
\pgfsetdash{}{0pt}
\pgfsetdash{}{0pt}
\pgfsetmiterjoin
\definecolor{dialinecolor}{rgb}{0.000000, 0.000000, 0.000000}
\pgfsetstrokecolor{dialinecolor}
\draw (0.345703\du,4.373779\du)--(0.345703\du,5.373779\du)--(5.673659\du,5.373779\du)--(5.673659\du,4.373779\du)--cycle;
\pgfsetlinewidth{0.050000\du}
\pgfsetdash{}{0pt}
\pgfsetdash{}{0pt}
\pgfsetmiterjoin
\definecolor{dialinecolor}{rgb}{0.000000, 0.000000, 0.000000}
\pgfsetstrokecolor{dialinecolor}
\draw (0.345703\du,2.898779\du)--(0.345703\du,3.898779\du)--(5.673659\du,3.898779\du)--(5.673659\du,2.898779\du)--cycle;
% setfont left to latex
\definecolor{dialinecolor}{rgb}{0.000000, 0.000000, 0.000000}
\pgfsetstrokecolor{dialinecolor}
\node at (3.009681\du,3.398779\du){ESP (Vojta)};
% setfont left to latex
\definecolor{dialinecolor}{rgb}{0.000000, 0.000000, 0.000000}
\pgfsetstrokecolor{dialinecolor}
\node at (3.009681\du,4.873779\du){ESP (out)};
% setfont left to latex
\definecolor{dialinecolor}{rgb}{0.000000, 0.000000, 0.000000}
\pgfsetstrokecolor{dialinecolor}
\node at (11.109944\du,1.748051\du){OpenWeather};
% setfont left to latex
\definecolor{dialinecolor}{rgb}{0.000000, 0.000000, 0.000000}
\pgfsetstrokecolor{dialinecolor}
\node at (11.105703\du,5.326279\du){broker2mongo.py};
% setfont left to latex
\definecolor{dialinecolor}{rgb}{0.000000, 0.000000, 0.000000}
\pgfsetstrokecolor{dialinecolor}
\node at (11.915703\du,8.476279\du){engine};
% setfont left to latex
\definecolor{dialinecolor}{rgb}{0.000000, 0.000000, 0.000000}
\pgfsetstrokecolor{dialinecolor}
\node at (19.025703\du,8.476279\du){trénování NN};
\pgfsetlinewidth{0.050000\du}
\pgfsetdash{}{0pt}
\pgfsetdash{}{0pt}
\pgfsetmiterjoin
\pgfsetbuttcap
{
\definecolor{dialinecolor}{rgb}{0.000000, 0.000000, 0.000000}
\pgfsetfillcolor{dialinecolor}
% was here!!!
\pgfsetarrowsend{stealth}
\definecolor{dialinecolor}{rgb}{0.000000, 0.000000, 0.000000}
\pgfsetstrokecolor{dialinecolor}
\pgfpathmoveto{\pgfpoint{5.673659\du}{6.276279\du}}
\pgfpathcurveto{\pgfpoint{6.586659\du}{6.276279\du}}{\pgfpoint{7.193203\du}{5.928779\du}}{\pgfpoint{7.945703\du}{5.701279\du}}
\pgfusepath{stroke}
}
\pgfsetlinewidth{0.050000\du}
\pgfsetdash{}{0pt}
\pgfsetdash{}{0pt}
\pgfsetmiterjoin
\pgfsetbuttcap
{
\definecolor{dialinecolor}{rgb}{0.000000, 0.000000, 0.000000}
\pgfsetfillcolor{dialinecolor}
% was here!!!
\pgfsetarrowsend{stealth}
\definecolor{dialinecolor}{rgb}{0.000000, 0.000000, 0.000000}
\pgfsetstrokecolor{dialinecolor}
\pgfpathmoveto{\pgfpoint{5.673659\du}{4.873779\du}}
\pgfpathcurveto{\pgfpoint{6.370859\du}{4.873779\du}}{\pgfpoint{7.148203\du}{5.323779\du}}{\pgfpoint{7.948203\du}{5.323779\du}}
\pgfusepath{stroke}
}
\pgfsetlinewidth{0.050000\du}
\pgfsetdash{}{0pt}
\pgfsetdash{}{0pt}
\pgfsetmiterjoin
\pgfsetbuttcap
{
\definecolor{dialinecolor}{rgb}{0.000000, 0.000000, 0.000000}
\pgfsetfillcolor{dialinecolor}
% was here!!!
\pgfsetarrowsend{stealth}
\definecolor{dialinecolor}{rgb}{0.000000, 0.000000, 0.000000}
\pgfsetstrokecolor{dialinecolor}
\pgfpathmoveto{\pgfpoint{5.673659\du}{3.398779\du}}
\pgfpathcurveto{\pgfpoint{6.238059\du}{3.398779\du}}{\pgfpoint{7.298203\du}{4.773779\du}}{\pgfpoint{7.948203\du}{4.923779\du}}
\pgfusepath{stroke}
}
\pgfsetlinewidth{0.050000\du}
\pgfsetdash{{1.000000\du}{1.000000\du}}{0\du}
\pgfsetdash{{0.150000\du}{0.150000\du}}{0\du}
\pgfsetbuttcap
{
\definecolor{dialinecolor}{rgb}{0.000000, 0.000000, 0.000000}
\pgfsetfillcolor{dialinecolor}
% was here!!!
\pgfsetarrowsstart{stealth}
\pgfsetarrowsend{stealth}
\definecolor{dialinecolor}{rgb}{0.000000, 0.000000, 0.000000}
\pgfsetstrokecolor{dialinecolor}
\draw (11.107209\du,2.505357\du)--(11.105800\du,4.676279\du);
}
% setfont left to latex
\definecolor{dialinecolor}{rgb}{0.000000, 0.000000, 0.000000}
\pgfsetstrokecolor{dialinecolor}
\node at (3.019681\du,6.276279\du){Home Assistant};
% setfont left to latex
\definecolor{dialinecolor}{rgb}{0.000000, 0.000000, 0.000000}
\pgfsetstrokecolor{dialinecolor}
\node at (5.088203\du,9.288779\du){GUI};
\pgfsetlinewidth{0.050000\du}
\pgfsetdash{}{0pt}
\pgfsetdash{}{0pt}
\pgfsetmiterjoin
\definecolor{dialinecolor}{rgb}{0.000000, 0.000000, 0.000000}
\pgfsetstrokecolor{dialinecolor}
\draw (0.365703\du,5.776279\du)--(0.365703\du,6.776279\du)--(5.673659\du,6.776279\du)--(5.673659\du,5.776279\du)--cycle;
% setfont left to latex
\definecolor{dialinecolor}{rgb}{0.000000, 0.000000, 0.000000}
\pgfsetstrokecolor{dialinecolor}
\node at (19.013124\du,5.326279\du){MongoDB};
\pgfsetlinewidth{0.050000\du}
\pgfsetdash{}{0pt}
\pgfsetdash{}{0pt}
\pgfsetmiterjoin
\definecolor{dialinecolor}{rgb}{0.000000, 0.000000, 0.000000}
\pgfsetstrokecolor{dialinecolor}
\draw (17.048124\du,4.826279\du)--(17.048124\du,5.826279\du)--(20.978124\du,5.826279\du)--(20.978124\du,4.826279\du)--cycle;
\pgfsetlinewidth{0.050000\du}
\pgfsetdash{{1.000000\du}{0.400000\du}{0.200000\du}{0.400000\du}}{0cm}
\pgfsetdash{{0.200000\du}{0.080000\du}{0.040000\du}{0.080000\du}}{0cm}
\pgfsetmiterjoin
\pgfsetbuttcap
{
\definecolor{dialinecolor}{rgb}{0.000000, 0.000000, 0.000000}
\pgfsetfillcolor{dialinecolor}
% was here!!!
\pgfsetarrowsend{stealth}
\definecolor{dialinecolor}{rgb}{0.000000, 0.000000, 0.000000}
\pgfsetstrokecolor{dialinecolor}
\pgfpathmoveto{\pgfpoint{14.095703\du}{5.326279\du}}
\pgfpathcurveto{\pgfpoint{15.065143\du}{5.326279\du}}{\pgfpoint{15.928684\du}{5.326279\du}}{\pgfpoint{16.898124\du}{5.326279\du}}
\pgfusepath{stroke}
}
\pgfsetlinewidth{0.050000\du}
\pgfsetdash{{0.200000\du}{0.080000\du}{0.040000\du}{0.080000\du}}{0cm}
\pgfsetdash{{0.200000\du}{0.080000\du}{0.040000\du}{0.080000\du}}{0cm}
\pgfsetbuttcap
{
\definecolor{dialinecolor}{rgb}{0.000000, 0.000000, 0.000000}
\pgfsetfillcolor{dialinecolor}
% was here!!!
\pgfsetarrowsend{stealth}
\definecolor{dialinecolor}{rgb}{0.000000, 0.000000, 0.000000}
\pgfsetstrokecolor{dialinecolor}
\draw (19.013124\du,5.826279\du)--(19.024826\du,7.826281\du);
}
\pgfsetlinewidth{0.050000\du}
\pgfsetdash{{0.200000\du}{0.080000\du}{0.040000\du}{0.080000\du}}{0cm}
\pgfsetdash{{0.200000\du}{0.080000\du}{0.040000\du}{0.080000\du}}{0cm}
\pgfsetmiterjoin
\pgfsetbuttcap
{
\definecolor{dialinecolor}{rgb}{0.000000, 0.000000, 0.000000}
\pgfsetfillcolor{dialinecolor}
% was here!!!
\pgfsetarrowsend{stealth}
\definecolor{dialinecolor}{rgb}{0.000000, 0.000000, 0.000000}
\pgfsetstrokecolor{dialinecolor}
\pgfpathmoveto{\pgfpoint{19.013124\du}{5.826279\du}}
\pgfpathcurveto{\pgfpoint{18.980703\du}{6.539538\du}}{\pgfpoint{11.920915\du}{6.236976\du}}{\pgfpoint{11.916153\du}{7.826280\du}}
\pgfusepath{stroke}
}
\pgfsetlinewidth{0.050000\du}
\pgfsetdash{{0.200000\du}{0.200000\du}}{0\du}
\pgfsetdash{{0.150000\du}{0.150000\du}}{0\du}
\pgfsetmiterjoin
\pgfsetbuttcap
{
\definecolor{dialinecolor}{rgb}{0.000000, 0.000000, 0.000000}
\pgfsetfillcolor{dialinecolor}
% was here!!!
\pgfsetarrowsstart{stealth}
\pgfsetarrowsend{stealth}
\definecolor{dialinecolor}{rgb}{0.000000, 0.000000, 0.000000}
\pgfsetstrokecolor{dialinecolor}
\pgfpathmoveto{\pgfpoint{7.430696\du}{9.287270\du}}
\pgfpathcurveto{\pgfpoint{9.055316\du}{9.270925\du}}{\pgfpoint{8.447331\du}{8.460564\du}}{\pgfpoint{10.045709\du}{8.474931\du}}
\pgfusepath{stroke}
}
\pgfsetlinewidth{0.030000\du}
\pgfsetdash{}{0pt}
\pgfsetdash{}{0pt}
\pgfsetmiterjoin
\definecolor{dialinecolor}{rgb}{0.938039, 0.938039, 0.938039}
\pgfsetfillcolor{dialinecolor}
\fill (14.368243\du,0.419048\du)--(14.368243\du,3.679831\du)--(22.635569\du,3.679831\du)--(22.635569\du,0.419048\du)--cycle;
\definecolor{dialinecolor}{rgb}{0.000000, 0.000000, 0.000000}
\pgfsetstrokecolor{dialinecolor}
\draw (14.368243\du,0.419048\du)--(14.368243\du,3.679831\du)--(22.635569\du,3.679831\du)--(22.635569\du,0.419048\du)--cycle;
\pgfsetlinewidth{0.050000\du}
\pgfsetdash{{0.150000\du}{0.150000\du}}{0\du}
\pgfsetdash{{0.150000\du}{0.150000\du}}{0\du}
\pgfsetbuttcap
{
\definecolor{dialinecolor}{rgb}{0.000000, 0.000000, 0.000000}
\pgfsetfillcolor{dialinecolor}
% was here!!!
\definecolor{dialinecolor}{rgb}{0.000000, 0.000000, 0.000000}
\pgfsetstrokecolor{dialinecolor}
\draw (14.660031\du,1.766999\du)--(16.848440\du,1.766999\du);
}
% setfont left to latex
\definecolor{dialinecolor}{rgb}{0.000000, 0.000000, 0.000000}
\pgfsetstrokecolor{dialinecolor}
\node[anchor=west] at (17.010545\du,1.750788\du){HTTP};
% setfont left to latex
\definecolor{dialinecolor}{rgb}{0.000000, 0.000000, 0.000000}
\pgfsetstrokecolor{dialinecolor}
\node[anchor=west] at (17.010545\du,2.480258\du){MongoDB Wire};
\pgfsetlinewidth{0.050000\du}
\pgfsetdash{{0.150000\du}{0.060000\du}{0.030000\du}{0.060000\du}}{0cm}
\pgfsetdash{{0.200000\du}{0.080000\du}{0.040000\du}{0.080000\du}}{0cm}
\pgfsetbuttcap
{
\definecolor{dialinecolor}{rgb}{0.000000, 0.000000, 0.000000}
\pgfsetfillcolor{dialinecolor}
% was here!!!
\definecolor{dialinecolor}{rgb}{0.000000, 0.000000, 0.000000}
\pgfsetstrokecolor{dialinecolor}
\draw (16.832230\du,2.480258\du)--(14.676241\du,2.480258\du);
}
\pgfsetlinewidth{0.050000\du}
\pgfsetdash{}{0pt}
\pgfsetdash{}{0pt}
\pgfsetbuttcap
{
\definecolor{dialinecolor}{rgb}{0.000000, 0.000000, 0.000000}
\pgfsetfillcolor{dialinecolor}
% was here!!!
\definecolor{dialinecolor}{rgb}{0.000000, 0.000000, 0.000000}
\pgfsetstrokecolor{dialinecolor}
\draw (14.643820\du,3.193518\du)--(16.799809\du,3.193518\du);
}
% setfont left to latex
\definecolor{dialinecolor}{rgb}{0.000000, 0.000000, 0.000000}
\pgfsetstrokecolor{dialinecolor}
\node[anchor=west] at (17.010545\du,3.209728\du){MQTT};
% setfont left to latex
\definecolor{dialinecolor}{rgb}{0.000000, 0.000000, 0.000000}
\pgfsetstrokecolor{dialinecolor}
\node at (18.501906\du,0.843004\du){Legenda protokolů};
\pgfsetlinewidth{0.050000\du}
\pgfsetdash{}{0pt}
\pgfsetdash{}{0pt}
\pgfsetbuttcap
{
\definecolor{dialinecolor}{rgb}{0.000000, 0.000000, 0.000000}
\pgfsetfillcolor{dialinecolor}
% was here!!!
\definecolor{dialinecolor}{rgb}{0.000000, 0.000000, 0.000000}
\pgfsetstrokecolor{dialinecolor}
\pgfpathmoveto{\pgfpoint{10.765159\du}{0.979587\du}}
\pgfpatharc{342}{180}{0.791422\du and 0.791422\du}
\pgfusepath{stroke}
}
\pgfsetlinewidth{0.050000\du}
\pgfsetdash{}{0pt}
\pgfsetdash{}{0pt}
\pgfsetbuttcap
{
\definecolor{dialinecolor}{rgb}{0.000000, 0.000000, 0.000000}
\pgfsetfillcolor{dialinecolor}
% was here!!!
\definecolor{dialinecolor}{rgb}{0.000000, 0.000000, 0.000000}
\pgfsetstrokecolor{dialinecolor}
\pgfpathmoveto{\pgfpoint{11.770910\du}{0.778101\du}}
\pgfpatharc{322}{226}{0.784160\du and 0.784160\du}
\pgfusepath{stroke}
}
\pgfsetlinewidth{0.050000\du}
\pgfsetdash{}{0pt}
\pgfsetdash{}{0pt}
\pgfsetbuttcap
{
\definecolor{dialinecolor}{rgb}{0.000000, 0.000000, 0.000000}
\pgfsetfillcolor{dialinecolor}
% was here!!!
\definecolor{dialinecolor}{rgb}{0.000000, 0.000000, 0.000000}
\pgfsetstrokecolor{dialinecolor}
\pgfpathmoveto{\pgfpoint{12.987050\du}{1.449327\du}}
\pgfpatharc{369}{217}{0.748755\du and 0.748755\du}
\pgfusepath{stroke}
}
\pgfsetlinewidth{0.050000\du}
\pgfsetdash{}{0pt}
\pgfsetdash{}{0pt}
\pgfsetbuttcap
{
\definecolor{dialinecolor}{rgb}{0.000000, 0.000000, 0.000000}
\pgfsetfillcolor{dialinecolor}
% was here!!!
\definecolor{dialinecolor}{rgb}{0.000000, 0.000000, 0.000000}
\pgfsetstrokecolor{dialinecolor}
\draw (8.846423\du,2.354572\du)--(13.368191\du,2.356142\du);
}
\pgfsetlinewidth{0.050000\du}
\pgfsetdash{}{0pt}
\pgfsetdash{}{0pt}
\pgfsetbuttcap
{
\definecolor{dialinecolor}{rgb}{0.000000, 0.000000, 0.000000}
\pgfsetfillcolor{dialinecolor}
% was here!!!
\definecolor{dialinecolor}{rgb}{0.000000, 0.000000, 0.000000}
\pgfsetstrokecolor{dialinecolor}
\pgfpathmoveto{\pgfpoint{9.233931\du}{1.242084\du}}
\pgfpatharc{289}{110}{0.589300\du and 0.589300\du}
\pgfusepath{stroke}
}
\pgfsetlinewidth{0.050000\du}
\pgfsetdash{}{0pt}
\pgfsetdash{}{0pt}
\pgfsetbuttcap
{
\definecolor{dialinecolor}{rgb}{0.000000, 0.000000, 0.000000}
\pgfsetfillcolor{dialinecolor}
% was here!!!
\definecolor{dialinecolor}{rgb}{0.000000, 0.000000, 0.000000}
\pgfsetstrokecolor{dialinecolor}
\pgfpathmoveto{\pgfpoint{13.358467\du}{2.366585\du}}
\pgfpatharc{420}{260}{0.542659\du and 0.542659\du}
\pgfusepath{stroke}
}
\end{tikzpicture}

    \caption[Schéma komunikace komponent]{Schéma komunikace komponent sytému automatického řízení žaluzií, tok dat mezi nimi a komunikační protokoly.}
    \label{fig:comm}
  \end{figure}

  Data se získávají ze 3 hlavních zdrojů na základě požadavku zaslaného skriptem \file{mqttDataCollector.py}. Stav žaluzie a informaci, zda je uživatel přítomen v~domácnosti\footnote{příznaky \code{posi\-tion}, \code{tilt} a \code{home}} odesílá systém domácí automatizace \emph{Home Assistant}, měřené veličiny (teplota a intenzita osvětlení) uvnitř a venku pak příslušná zařízení \footnote{příznaky \code{lum\_in}, \code{lum\_out}, \code{temp\_in} a \code{temp\_out}} a ostatní informace o~počasí\footnote{rychlost (\code{owm\_wind\_speed}) a směr (\code{owm\_wind\_heading}) větru, předpověď teploty na $x=1,2,3$~h dopředu (\code{owm\_temp\_$x$h}) a předpověď nejvyšší denní teploty (\code{owm\_temp\_max})} se získávají přímo v~rámci skriptu \file{mqttData\-Collector.py} z~REST API \href{https://openweathermap.org/}{OpenWeather}. Data získaná z~komponent se pak v~jedné společné zprávě odesílají přes MQTT skriptu \file{broker2mongo.py}, který je uloží do databáze. Na \refskl{fig:comm}{obrázku} je přehled komponent, které se podílejí na sběru dat, a jejich komunikace včetně používaných protokolů.
  \section{Použití komunikačního protokolu MQTT} \label{sec:MQTT}
    Protokol MQTT (\cref{ssec:mqtt}) % TODO možná vynechat referenci
    se využívá pro veškerou komunikaci součástí při sběru dat. Centrální uzel se nazývá broker a jedná se o~software spuštěný na Raspberry Pi (\cref{list:rpi}) na výchozím TCP portu 1883. Připojují se k~němu všechny komponenty, které využívají MQTT. Po připojení mohou publikovat zprávy do hirerchicky uspořádaných témat a přihlašovat se k~jejich odběru. Broker pak zajistí doručení zpráv publikovaných v~určitém tématu klientům, kteří jsou přihlášeni k~jeho odběru.
    
    Při sběru dat se periodicky nebo na základě změny stavu žaluzie odešle do tématu \code{smart\-blinds/command} zpráva ve formátu JSON, která obsahuje pod klíčem \uv{\emph{command}} hodnotu \uv{\emph{request\_values}}. K~odběru tohoto tématu jsou přihlášena obě zařízení s~ESP8266 (\cref{sec:senzory}) i systém domácí automatizace a po jejím přijetí odešlou aktuální hodnoty jimi sledovaných příznaků do odpovídajích témat dle \refskl{tab:topics}{tabulky}. Všechna tato témata odebírá skript \file{mqttDataCollec\-tor.py}, který hodnoty příznaků společně s~časovou známkou odesílá v~jedné zprávě do tématu \code{smart\-blinds/da\-ta}. Tato zpráva dále obsahuje informaci o~tom, zda je aktivní testovací režim využívaný při vývoji. Téma \code{smart\-blinds/da\-ta} odebírá skript \file{broker\-2mongo.py} popsaný v~\refskl{sec:db}{sekci}. Ten všechna takto přenesená data ukládá do databáze k pozdějšímu použití.
  % Table generated by Excel2LaTeX from sheet 'List1'
\begin{table}[htbp]
\centering
    \begin{tabular}{l|l}
    \toprule
    \multicolumn{1}{c|}{Příznak nebo stav} & \multicolumn{1}{c}{Téma} \\
    \midrule
    \midrule
    temp\_inside & smartblinds/temp/Vojta \\
    \midrule
    temp\_outside & smartblinds/temp/out \\
    \midrule
    lum\_inside & smartblinds/lux/Vojta \\
    \midrule
    lum\_outside & smartblinds/lux/out \\
    \midrule
    home  & smartblinds/presence/Vojta \\
    \midrule
    position & smartblinds/position/Vojta \\
    \midrule
    tilt  & smartblinds/tilt/Vojta \\
    \bottomrule
    \bottomrule
    \end{tabular}%
    \caption[Příznaky a jejich MQTT témata]{Názvy příznaků a MQTT témata, kam se hodnoty příznaků odesílají, a která odebírá skript \file{mqttDataCollector.py}.}
\label[]{tab:topics}
\end{table}%

  \section{Komunikace s pohonem žaluzie}
    S pohonem žaluzie komunikuje centrální jednotka Tahoma, dodávaná výrobcem žaluzie, pomocí bezdrátového proprietárního protokolu \emph{io}. Tato jednotka je dále připojená do internetu a je možné s ní komunikovat prostřednictvím \acrlong{api} (\acrshort{api}), které nabízí výrobce. Komunikace s~\acrshort{api} je šifrovaná a probíhá pomocí protokolu HTTPS zasíláním požadavků na \emph{endpointy} serverů výrobce dostupné na adrese \href{https://api.somfy.com/api/v1/}{https://api.somfy.com/api/v1/}. Použitý systém domácí automatizace i backend vyvíjeného systému používají ke komunikaci s~API knihovnu Pymfy, která mapuje akce pro manipulaci se žaluzií a její stav na metody a proměnné objektu v~jazyce Python, který je tak modelem žaluzie. Dále zpřístupňuje některé obecnější metody pro práci s~API jako je například zjištění všech montáží zákazníka, zjištění všech dostupných zařízení v~rámci konkrétní montáže atd.

    Jednotlivé požadavky se autorizují na základě krátkodobého tokenu s~platností 1~h. Pokud vyprší jeho platnost, je nutné pomocí obnovovacího tokenu ze serveru získat nový a přikládat ho k~budoucím požadavkům. Oba tokeny se získávají pomocí OAuth2 metodou \uv{Authorization Code Grant}, z~důvodů odlišností od dokumentace v~implementaci Somfy se ale ani po kontaktování technické podpory nepodařilo získat nové přístupové údaje k~API a tak muselo být využito nedokumentovaného postupu k~jejich získání. Přestože jsou tokeny v~obou systémech odvozené od stejných přístupových údajů, zdají se být nezávislé (včetně kvóty na četnost požadavků) a na funkčnost to tedy nemá vliv. Hledání tohoto postupu se zdálo být časově nákladné a proto se ke zjišťování aktuálního stavu žaluzií využívá právě systém domácí automatizace.

    Běžně je nutné si na webových stránkách na adrese \href{https://develeoper.somfy.com}{https://developer.som\-fy.com} vytvořit tzv. aplikaci pod uživatelským účtem, ke kterému je technikem při instalaci žaluzií přiřazena konkrétní montáž. Na základě zadaného názvu, \emph{callback \acrshort{url}}\footnote[1]{Návratová \acrshort{url} v~rámci vyvíjené aplikace, na kterou se přesměruje webový prohlížeč uživatele po úspěšné autorizaci, prostřednictvím parametrů se aplikaci předá kód, na základě kterého může získat tokeny} a popisu aplikace se získájí dva řetězce: \emph{Consumer key} a \emph{Consumer secret}. (\cite{somfy:api}) Ty se pak společně s~\emph{callback \acrshort{url}}\footnotemark[1] předají konstruktoru objektu, který v~rámci knihovny Pymfy reprezentuje API. (\cite{tetienne:pymfy})

    Takto vytvořené \emph{Consumer key} a \emph{Consumer secret} se však nedařilo použít, server Somfy je totiž zamítal. Byla tedy kontaktována technická podpora, ale ani po několika týdnech se nedostavila odpověď. Mezitím pokračovaly pokusy o~získání tokenu jinak. Pro systém domácí automatizace, který se se žaluziemi již používal, existovaly tyto údaje a ukázalo se, že jsou funkční. Místo správné \emph{callback \acrshort{url}} se tedy do konstruktoru zadala ta, která příslušela k~aplikaci ve vývojářském portálu, a byl zahájen proces získání tokenů. Po přesměrování zpět se v~adresním řádku prohlížeče ručně změnila adresa tak, aby odpovídala endpointu vytvořenému k~získávání a ukládání tokenů v~rámci backendu. Pokud by bylo možné postupovat standardním způsobem, po přihlášení na stránkách Somfy by byl prohlížeč přesměrován právě na tuto adresu. Tokeny se tak uložily do souboru \file{somfycache} pro pozdější použití.

    Data a služby, které API poskytuje jsou využívány ke dvěma účelům. Jednak systém domácí automatizace každou minutu kontroluje aktuální stav žaluzií a pokud se změní, odešle novou hodnotu přes MQTT a spustí tak posloupnost akcí (popsanou v~\refskl{sec:MQTT}{sekci}), které vedou k~vytvoření nového záznamu v~databázi, jednak stránka \uv{Control} v~\acrshort{gui} umožňuje zobrazit aktuální stav žaluzií a zadávat příkazy k~jeho změně. Samotnou komunikaci s~API zajišťuje backend systému, se kterým se komunikuje přes protokol WebSockets (\cref{ssec:ws}). Po připojení alespoň jednoho klienta se každých 15~s~zjistí stav žaluzie a připojeným prohlížečům se odešle zpráva ve formátu JSON. Její struktura je uvedená v~\refskl{lst:wsMsg}{úryvku kódu}. Naopak po přijetí zprávy backendem se v~závislosti na jejím obsahu odešle požadavek API na změnu stavu žaluzie. Zpráva má stejnou strukturu jako zpráva v~\refskl{lst:wsMsg}{úryvku kódu}, ale obsahuje jen jeden z~klíčů. Navíc může obsahovat speciální klíč \code{testing} s hodnotou datového typu \code{boolean}, pomocí kterého lze měnit příznak \code{testing} dat ukládaných do databáze, který slouží k~rozlišení akcí vyvolaných při vývoji v~rámci testování a skutečných akcí, které má systém napodobovat. Zpráva o~změně režimu se v~tomto případě odešle přes MQTT skriptu \file{mqttDataCollector.py}, který příznak zaznamenává k~datům. Zprávy se odesílají na základě interakce uživatele s~ovládacími prvky \acrshort{gui}.

    \begin{lstlisting}[caption={[Struktura zprávy o~stavu žaluzie]Struktura zprávy o~stavu žaluzie, která je všem klientům zasílána každých 15~s. Klíč \code{position} označuje výšku vytažení žaluzie (0 -- zavřeno, 100 -- otevřeno), obdobně hodnota \code{tilt} vyjadřuje naklopení lamel.},captionpos=b,label=lst:wsMsg]
      {
        "position": int
        "tilt": int
      }
    \end{lstlisting}

  \section{Využití služby OpenWeather} \label{sec:owm}
    Skript \file{mqttDataCollector.py} kromě měřených příznaků a stavu žaluzií zjišťuje také odhad počasí a jeho předpověď. Tato data poskytuje společnost OpenWeather prostřednictvím svého API, které má několik možností využití. Z~důvodu jednoduchosti byla pro přístup k~API zvolena knihovna PyOWM, která používá variantu \uv{One-Call}. Na základě jednoho požadavku se tak získá aktuální počasí, předpověď po minutách na následující hodinu, předpověď po hodinách na následující 2 dny, předpověď po dnech na následující týden, výstrahy vydané Českým hydrometeorologickým ústavem a historická data z~posledních 5 dnů, vše je pak přístupné pomocí atributů a metod objektu v~Pythonu. K~požadavku je vždy připojen klíč, který byl bezplatně získán po registraci na \href{https://openweathermap.org}{webových stránkách https://openweathermap.org}. Limity četnosti požadavků stanovené poskytovatelem jsou vyšší než 1 požadavek za 5 minut. Jako příznaků se využívá kódu počasí (vyjadřuje jeho shrnutí -- např. slunečno, polojasno, déšť, jasná noc atp.), rychlosti a směru větru, předpovědi teploty na následující 3 hodiny a předpovědi nejvyšší denní teploty.

  \section{Databáze}\label{sec:db}
    Data, která se sbírají pomocí skriptu \file{mqttDataCollector.py} v~pětiminutových intervalech nebo na základě změny stavu žaluzie uživatelem, ukládá skript \file{broker2mongo.py} do databáze MongoDB (\cite{mongodb:mongodb}). Jednotlivé vzorky jsou získávány z~MQTT zpráv zasílaných do tématu \code{smart\-blinds/data} a ukládají se ve stejné podobě jako příchozí zpráva. Struktura je uvedena v~\refskl{lst:datum}{úryvku kódu}.

    \begin{lstlisting}[caption={[Struktura vzorku v~databázi]Struktura vzorku dat uloženého jako záznam v~databázi MongoDB. Obsahuje hodnoty příznaků i stav žaluzií v~okamžiku jeho pořízení a časovou známku. Místo hodnot jsou zde uvedeny jejich datové typy (\code{float} - číslo s plovoucí řádovou čárkou, \code{boolean} - pravdivostní hodnota, \code{int} - celé číslo)},captionpos=b,label=lst:datum]
      {
        "timestamp": float,
        "testing": boolean,   
        "periodical": boolean,
        "features": {
          "year_day": int,
          "week_day": int,
          "day_secs": int,
          "home": boolean,
          "temp_in": float,
          "temp_out": float,
          "lum_in": float,
          "lum_out": float,
          "owm_temp_max": float,
          "owm_temp_1h": float,
          "owm_temp_2h": float,
          "owm_temp_3h": float,
          "owm_code": int,
          "owm_wind_speed": float,
          "owm_wind_heading": float
        },
        "targets": {
          "position": int,         
          "tilt": int
        }
      }
    \end{lstlisting}

    Uložená data se využívají při strojovém učení modelů, ve vizualizaci dat na stránce \uv{Live} v~\acrshort{gui} a v~porovnání skutečného řízení a řízení jednotlivých regresorů tamtéž. Také se zobrazují v~tabulce na stránce \uv{Data}. Vyhodnocení sběru dat je uvedené v \refskl{sec:resData}{sekci}.

  \section{Webový server pro komunikaci s \acrshort{gui}} \label{sec:tornado}
    Systém má webové uživatelské rozhraní poskytované \emph{Next.js} serverem (\cref{chap:gui}). Jeho propojení s jádrem systému (backend) je řešeno pomocí frameworku Tornado v jazyce Python. Tento webserver tak tvoří rozhraní mezi \acrshort{gui} a databází, všemi regresory (predikce, simulace i učení) a Somfy API. Klient v podobě webového prohlížeče odesílá HTTP požadavky na tyto endpointy: 
    \begin{itemize}
      \item \code{ep\_data} poskytuje data z~databáze na základě \code{POST} požadavku s~volitelnými JSON parametry (\code{ts\_start} a \code{ts\_end}) v~těle označujícími časovou známku začátku a konce intervalu, vrací seznam vzorků (\cref{lst:datum}) pod klíčem \code{data} ve formátu JSON jako tělo odpovědi.
      \item \code{ep\_control} slouží k~predikci řízení jednoho z~regresorů (parametr \code{classifier\_name} na základě hodnot příznaků předaných jako JSON parametr \code{features} (jeden vzorek), vrací \code{con\-trol\-Time} - dobu trvání predikce, \code{status: ok} a navržené hodnoty řízení (jako v~\refskl{lst:wsMsg}{úryvku kódu}).
      \item \code{ep\_train} spustí přetrénování \acrshort{nn} (\cref{sec:retraining}) na všech dostupných datech.
      \item \code{ep\_predict} slouží k predikci řízení regresorů předaných v~parametru \code{classifiers} pro reálná data (přijímá začátek a konec intervalu stejným způsobem jako \code{ep\_data}, načte všechny vzorky z tohoto interavlu). Pro každý z regresorů vrací pod klíčem s~jejich názvem dobu predikce a seznam dvojic doporučeného řízení a odpovídajích časových známek.
      \item \code{ws} zajišťuje spojení pomocí protokolu WebSockets (\cref{ssec:ws}) pro pravidelnou aktualizaci vizualizace stavu žaluzií a pro přenos požadavků na jeho změnu.
    \end{itemize}
    Schéma těchto endpointů a některých přenášených dat je na \refskl{fig:endpoints}{obrázku}.
    
    \begin{figure}[h]
      \centering
      \begin{adjustbox}{width=\textwidth}
        % Graphic for TeX using PGF
% Title: C:\Users\vojta\Documents\Škola\bakalarka\iot\egs\smartblinds\doc\img\endpoints.dia
% Creator: Dia v0.97.2
% CreationDate: Thu Mar 31 13:54:07 2022
% For: vojta
% \usepackage{tikz}
% The following commands are not supported in PSTricks at present
% We define them conditionally, so when they are implemented,
% this pgf file will use them.
\ifx\du\undefined
  \newlength{\du}
\fi
\setlength{\du}{15\unitlength}
\begin{tikzpicture}
\pgftransformxscale{1.000000}
\pgftransformyscale{-1.000000}
\definecolor{dialinecolor}{rgb}{0.000000, 0.000000, 0.000000}
\pgfsetstrokecolor{dialinecolor}
\definecolor{dialinecolor}{rgb}{1.000000, 1.000000, 1.000000}
\pgfsetfillcolor{dialinecolor}
\pgfsetlinewidth{0.050000\du}
\pgfsetdash{}{0pt}
\pgfsetdash{}{0pt}
\pgfsetmiterjoin
\definecolor{dialinecolor}{rgb}{1.000000, 1.000000, 1.000000}
\pgfsetfillcolor{dialinecolor}
\fill (0.600000\du,4.600000\du)--(0.600000\du,6.350000\du)--(5.200000\du,6.350000\du)--(5.200000\du,4.600000\du)--cycle;
\definecolor{dialinecolor}{rgb}{0.000000, 0.000000, 0.000000}
\pgfsetstrokecolor{dialinecolor}
\draw (0.600000\du,4.600000\du)--(0.600000\du,6.350000\du)--(5.200000\du,6.350000\du)--(5.200000\du,4.600000\du)--cycle;
% setfont left to latex
\definecolor{dialinecolor}{rgb}{0.000000, 0.000000, 0.000000}
\pgfsetstrokecolor{dialinecolor}
\node at (2.900000\du,5.475000\du){ep\_train};
\pgfsetlinewidth{0.050000\du}
\pgfsetdash{}{0pt}
\pgfsetdash{}{0pt}
\pgfsetmiterjoin
\definecolor{dialinecolor}{rgb}{1.000000, 1.000000, 1.000000}
\pgfsetfillcolor{dialinecolor}
\fill (0.600000\du,0.250000\du)--(0.600000\du,2.000000\du)--(5.200000\du,2.000000\du)--(5.200000\du,0.250000\du)--cycle;
\definecolor{dialinecolor}{rgb}{0.000000, 0.000000, 0.000000}
\pgfsetstrokecolor{dialinecolor}
\draw (0.600000\du,0.250000\du)--(0.600000\du,2.000000\du)--(5.200000\du,2.000000\du)--(5.200000\du,0.250000\du)--cycle;
% setfont left to latex
\definecolor{dialinecolor}{rgb}{0.000000, 0.000000, 0.000000}
\pgfsetstrokecolor{dialinecolor}
\node at (2.900000\du,1.125000\du){ep\_data};
\pgfsetlinewidth{0.050000\du}
\pgfsetdash{}{0pt}
\pgfsetdash{}{0pt}
\pgfsetmiterjoin
\definecolor{dialinecolor}{rgb}{1.000000, 1.000000, 1.000000}
\pgfsetfillcolor{dialinecolor}
\fill (0.600000\du,2.379155\du)--(0.600000\du,4.129155\du)--(5.200000\du,4.129155\du)--(5.200000\du,2.379155\du)--cycle;
\definecolor{dialinecolor}{rgb}{0.000000, 0.000000, 0.000000}
\pgfsetstrokecolor{dialinecolor}
\draw (0.600000\du,2.379155\du)--(0.600000\du,4.129155\du)--(5.200000\du,4.129155\du)--(5.200000\du,2.379155\du)--cycle;
% setfont left to latex
\definecolor{dialinecolor}{rgb}{0.000000, 0.000000, 0.000000}
\pgfsetstrokecolor{dialinecolor}
\node at (2.900000\du,3.254155\du){ep\_control};
% setfont left to latex
\definecolor{dialinecolor}{rgb}{0.000000, 0.000000, 0.000000}
\pgfsetstrokecolor{dialinecolor}
\node[anchor=west] at (8.837802\du,0.625859\du){ \{ts\_start: int, ts\_end: int\}};
\pgfsetlinewidth{0.050000\du}
\pgfsetdash{}{0pt}
\pgfsetdash{}{0pt}
\pgfsetmiterjoin
\definecolor{dialinecolor}{rgb}{1.000000, 1.000000, 1.000000}
\pgfsetfillcolor{dialinecolor}
\fill (0.600000\du,6.775426\du)--(0.600000\du,8.525426\du)--(5.200000\du,8.525426\du)--(5.200000\du,6.775426\du)--cycle;
\definecolor{dialinecolor}{rgb}{0.000000, 0.000000, 0.000000}
\pgfsetstrokecolor{dialinecolor}
\draw (0.600000\du,6.775426\du)--(0.600000\du,8.525426\du)--(5.200000\du,8.525426\du)--(5.200000\du,6.775426\du)--cycle;
% setfont left to latex
\definecolor{dialinecolor}{rgb}{0.000000, 0.000000, 0.000000}
\pgfsetstrokecolor{dialinecolor}
\node at (2.900000\du,7.650426\du){ep\_predict};
\pgfsetlinewidth{0.050000\du}
\pgfsetdash{}{0pt}
\pgfsetdash{}{0pt}
\pgfsetmiterjoin
\definecolor{dialinecolor}{rgb}{1.000000, 1.000000, 1.000000}
\pgfsetfillcolor{dialinecolor}
\fill (0.600000\du,8.931863\du)--(0.600000\du,10.681863\du)--(5.200000\du,10.681863\du)--(5.200000\du,8.931863\du)--cycle;
\definecolor{dialinecolor}{rgb}{0.000000, 0.000000, 0.000000}
\pgfsetstrokecolor{dialinecolor}
\draw (0.600000\du,8.931863\du)--(0.600000\du,10.681863\du)--(5.200000\du,10.681863\du)--(5.200000\du,8.931863\du)--cycle;
% setfont left to latex
\definecolor{dialinecolor}{rgb}{0.000000, 0.000000, 0.000000}
\pgfsetstrokecolor{dialinecolor}
\node at (2.900000\du,9.806863\du){ws};
\pgfsetlinewidth{0.050000\du}
\pgfsetdash{}{0pt}
\pgfsetdash{}{0pt}
\pgfsetbuttcap
{
\definecolor{dialinecolor}{rgb}{0.000000, 0.000000, 0.000000}
\pgfsetfillcolor{dialinecolor}
% was here!!!
\pgfsetarrowsend{stealth}
\definecolor{dialinecolor}{rgb}{0.000000, 0.000000, 0.000000}
\pgfsetstrokecolor{dialinecolor}
\draw (5.200000\du,1.615616\du)--(8.800000\du,1.615616\du);
}
\pgfsetlinewidth{0.050000\du}
\pgfsetdash{}{0pt}
\pgfsetdash{}{0pt}
\pgfsetbuttcap
{
\definecolor{dialinecolor}{rgb}{0.000000, 0.000000, 0.000000}
\pgfsetfillcolor{dialinecolor}
% was here!!!
\pgfsetarrowsend{stealth}
\definecolor{dialinecolor}{rgb}{0.000000, 0.000000, 0.000000}
\pgfsetstrokecolor{dialinecolor}
\draw (8.800000\du,0.615616\du)--(5.200000\du,0.615616\du);
}
% setfont left to latex
\definecolor{dialinecolor}{rgb}{0.000000, 0.000000, 0.000000}
\pgfsetstrokecolor{dialinecolor}
\node[anchor=west] at (8.814753\du,1.483064\du){ \{data: \ensuremath{[}...\ensuremath{]}\}};
% setfont left to latex
\definecolor{dialinecolor}{rgb}{0.000000, 0.000000, 0.000000}
\pgfsetstrokecolor{dialinecolor}
\node[anchor=west] at (8.803545\du,2.835323\du){ \{classifier\_name: str, features: \{year\_day: int, ...\}\}};
% setfont left to latex
\definecolor{dialinecolor}{rgb}{0.000000, 0.000000, 0.000000}
\pgfsetstrokecolor{dialinecolor}
\node[anchor=west] at (8.803545\du,3.692528\du){ \{status: str, targets: \{position: int, tilt: int\}, controlTime: float\}};
% setfont left to latex
\definecolor{dialinecolor}{rgb}{0.000000, 0.000000, 0.000000}
\pgfsetstrokecolor{dialinecolor}
\node[anchor=west] at (8.819453\du,4.862799\du){ };
% setfont left to latex
\definecolor{dialinecolor}{rgb}{0.000000, 0.000000, 0.000000}
\pgfsetstrokecolor{dialinecolor}
\node[anchor=west] at (8.819453\du,5.848964\du){ \{status: str, classifierInfo: \{ffnn: \{lastTrained: int, ...\}, ...\}\}};
% setfont left to latex
\definecolor{dialinecolor}{rgb}{0.000000, 0.000000, 0.000000}
\pgfsetstrokecolor{dialinecolor}
\node[anchor=west] at (8.812466\du,7.201224\du){ \{ts\_start: int, ts\_end: int, classifiers: str\ensuremath{[}\ensuremath{]}\}};
% setfont left to latex
\definecolor{dialinecolor}{rgb}{0.000000, 0.000000, 0.000000}
\pgfsetstrokecolor{dialinecolor}
\node[anchor=west] at (8.812466\du,8.040753\du){ \{status: str, ffnn: \ensuremath{[}int\ensuremath{[}\ensuremath{]}, \ensuremath{[}float\ensuremath{[}\ensuremath{]}, float\ensuremath{[}\ensuremath{]}\ensuremath{]}\ensuremath{]}, ...\}};
% setfont left to latex
\definecolor{dialinecolor}{rgb}{0.000000, 0.000000, 0.000000}
\pgfsetstrokecolor{dialinecolor}
\node[anchor=west] at (8.810698\du,9.357661\du){ \{position: int\}, \{tilt: int\}, \{testing: bool\}};
% setfont left to latex
\definecolor{dialinecolor}{rgb}{0.000000, 0.000000, 0.000000}
\pgfsetstrokecolor{dialinecolor}
\node[anchor=west] at (8.810698\du,10.214865\du){ \{position: int, tilt: int\}};
% setfont left to latex
\definecolor{dialinecolor}{rgb}{0.000000, 0.000000, 0.000000}
\pgfsetstrokecolor{dialinecolor}
\node at (7.034879\du,1.129549\du){POST};
\pgfsetlinewidth{0.050000\du}
\pgfsetdash{}{0pt}
\pgfsetdash{}{0pt}
\pgfsetbuttcap
{
\definecolor{dialinecolor}{rgb}{0.000000, 0.000000, 0.000000}
\pgfsetfillcolor{dialinecolor}
% was here!!!
\pgfsetarrowsend{stealth}
\definecolor{dialinecolor}{rgb}{0.000000, 0.000000, 0.000000}
\pgfsetstrokecolor{dialinecolor}
\draw (5.215388\du,3.814157\du)--(8.815388\du,3.814157\du);
}
\pgfsetlinewidth{0.050000\du}
\pgfsetdash{}{0pt}
\pgfsetdash{}{0pt}
\pgfsetbuttcap
{
\definecolor{dialinecolor}{rgb}{0.000000, 0.000000, 0.000000}
\pgfsetfillcolor{dialinecolor}
% was here!!!
\pgfsetarrowsend{stealth}
\definecolor{dialinecolor}{rgb}{0.000000, 0.000000, 0.000000}
\pgfsetstrokecolor{dialinecolor}
\draw (8.815388\du,2.814157\du)--(5.215388\du,2.814157\du);
}
% setfont left to latex
\definecolor{dialinecolor}{rgb}{0.000000, 0.000000, 0.000000}
\pgfsetstrokecolor{dialinecolor}
\node at (7.050268\du,3.328090\du){POST};
\pgfsetlinewidth{0.050000\du}
\pgfsetdash{}{0pt}
\pgfsetdash{}{0pt}
\pgfsetbuttcap
{
\definecolor{dialinecolor}{rgb}{0.000000, 0.000000, 0.000000}
\pgfsetfillcolor{dialinecolor}
% was here!!!
\pgfsetarrowsend{stealth}
\definecolor{dialinecolor}{rgb}{0.000000, 0.000000, 0.000000}
\pgfsetstrokecolor{dialinecolor}
\draw (5.195945\du,5.988269\du)--(8.795945\du,5.988269\du);
}
\pgfsetlinewidth{0.050000\du}
\pgfsetdash{}{0pt}
\pgfsetdash{}{0pt}
\pgfsetbuttcap
{
\definecolor{dialinecolor}{rgb}{0.000000, 0.000000, 0.000000}
\pgfsetfillcolor{dialinecolor}
% was here!!!
\pgfsetarrowsend{stealth}
\definecolor{dialinecolor}{rgb}{0.000000, 0.000000, 0.000000}
\pgfsetstrokecolor{dialinecolor}
\draw (8.795945\du,4.988269\du)--(5.195945\du,4.988269\du);
}
% setfont left to latex
\definecolor{dialinecolor}{rgb}{0.000000, 0.000000, 0.000000}
\pgfsetstrokecolor{dialinecolor}
\node at (7.030824\du,5.502202\du){POST};
\pgfsetlinewidth{0.050000\du}
\pgfsetdash{}{0pt}
\pgfsetdash{}{0pt}
\pgfsetbuttcap
{
\definecolor{dialinecolor}{rgb}{0.000000, 0.000000, 0.000000}
\pgfsetfillcolor{dialinecolor}
% was here!!!
\pgfsetarrowsend{stealth}
\definecolor{dialinecolor}{rgb}{0.000000, 0.000000, 0.000000}
\pgfsetstrokecolor{dialinecolor}
\draw (5.211853\du,8.180057\du)--(8.811853\du,8.180057\du);
}
\pgfsetlinewidth{0.050000\du}
\pgfsetdash{}{0pt}
\pgfsetdash{}{0pt}
\pgfsetbuttcap
{
\definecolor{dialinecolor}{rgb}{0.000000, 0.000000, 0.000000}
\pgfsetfillcolor{dialinecolor}
% was here!!!
\pgfsetarrowsend{stealth}
\definecolor{dialinecolor}{rgb}{0.000000, 0.000000, 0.000000}
\pgfsetstrokecolor{dialinecolor}
\draw (8.811853\du,7.180057\du)--(5.211853\du,7.180057\du);
}
% setfont left to latex
\definecolor{dialinecolor}{rgb}{0.000000, 0.000000, 0.000000}
\pgfsetstrokecolor{dialinecolor}
\node at (7.046732\du,7.693991\du){POST};
\pgfsetlinewidth{0.050000\du}
\pgfsetdash{}{0pt}
\pgfsetdash{}{0pt}
\pgfsetbuttcap
{
\definecolor{dialinecolor}{rgb}{0.000000, 0.000000, 0.000000}
\pgfsetfillcolor{dialinecolor}
% was here!!!
\pgfsetarrowsend{stealth}
\definecolor{dialinecolor}{rgb}{0.000000, 0.000000, 0.000000}
\pgfsetstrokecolor{dialinecolor}
\draw (5.174734\du,10.318819\du)--(8.774734\du,10.318819\du);
}
\pgfsetlinewidth{0.050000\du}
\pgfsetdash{}{0pt}
\pgfsetdash{}{0pt}
\pgfsetbuttcap
{
\definecolor{dialinecolor}{rgb}{0.000000, 0.000000, 0.000000}
\pgfsetfillcolor{dialinecolor}
% was here!!!
\pgfsetarrowsend{stealth}
\definecolor{dialinecolor}{rgb}{0.000000, 0.000000, 0.000000}
\pgfsetstrokecolor{dialinecolor}
\draw (8.774734\du,9.318819\du)--(5.174734\du,9.318819\du);
}
% setfont left to latex
\definecolor{dialinecolor}{rgb}{0.000000, 0.000000, 0.000000}
\pgfsetstrokecolor{dialinecolor}
\node at (7.009613\du,9.832752\du){WS};
\end{tikzpicture}

      \end{adjustbox}
      \caption[Schéma endpointů a přenášených struktur]{Schéma endpointů backendu systému pro automatické ovládání žaluzií a datové struktury přenášené v~těle jednotlivých požadavků a odpovědí včetně metody. V~posledním případě se jedná o~protokol WebSockets.}
      \label{fig:endpoints}
    \end{figure}