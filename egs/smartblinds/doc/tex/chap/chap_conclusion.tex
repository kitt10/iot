\chapter{Závěr} \label{chap:conclusion}
% \begin{itemize}
%     \item the Introduction of the paper gives the current scientific context
%     \item the Discussion shows how your data lead to or support the ideas
%     \item the Conclusion summarizes the ideas in one paragraph
% \end{itemize}
% Skeleton:
% \begin{itemize}
%     \item Conclusion (one or two main ideas)
%     \item Outlook (what are you going to do next)  
% \end{itemize}


% \section{Future Work} \label{sec:future_work}
% Outlook...
    V rámci této práce byl vyvinut systém, který v reálném čase sbírá hodnoty příznaků a aktuální uživatelské nastavení žaluzie, pomocí 3 různých regresorů odhaduje vhodné nastavení žaluzie a umožňuje řízení generované jednotlivými regresory porovnat. \acrshort{nn} se v systému periodicky přetrénovávají (retraining), což má pozitivní vliv na přesnost jejich odhadu, vyskytnou-li se v datech nové skutečnosti.

    Hardwarové součásti použité v měřicích zařízeních se osvědčily, stejně jako datové spojení s ostatními součástmi systému. Bylo by vhodné monitorovat všechny součásti i jejich spojení a případné problémy nejlépe automaticky odstraňovat, nebo hlásit uživateli. V systému se uživateli hlásí jen to, když nepřijdou data z některého zdroje dat v okamžiku pořizování vzorku.

    V úloze automatického ovládání žaluzií může jejich stav záviset nejen na aktuálních podmínkách ale také na podmínkách, které panovaly v historii. Kvůli tomu je vhodné využít regresor, který historické podmínky uvažuje při svém odhadu. V této práci tuto vlastnost splňuje regresor založný na rekurentní neuronové síti s LSTM buňkami.

    Přesto, že některé veličiny mohou být silně svázány s odhadovanými, nemusí být dobrými příznaky pro reálné využití, pokud je odhad řízení při jeho použití zpětně ovlivňuje.

    Díky dalšímu sběru dat společně s přetrénováním \acrshort{nn} by regresory mohly do budoucna poskytovat postupně lepší výsledky. V případě skutečného nasazení by bylo nutné ošetřit konfliktní stavy uživatelem vyžádaného nastavení žaluzie a systémem odhadnutým nastavením tak, aby interakce se systémem byla pro uživatele přívětivá.