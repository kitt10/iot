\chapter{Úvod} \label{chap:introduction}
% \uv{Introduction section of your paper leads a reader from a well-known to the particular spot}

% Skeleton:
% \begin{itemize}
%     \item background
%     \item gap
%     \item proposed method
%     \item outline
% \end{itemize}

% motivace, zasazení do kontextu, o čem práce je, jaké jsou cíle práce, rozvržení práce (outline)
% \begin{itemize}
%     \item hlavní motivace: zapojení strojového učení na realtime problém
%     \item retraining na základě nově nasbíraných dat
%     \item naučení se návyků uživatele
%     \item naučení se nových příznaků v průběhu času (přes roční období…)
%     \item vyhodnocení vlivu zapojení kontextu na kvalitu predikce (LSTM vs. FFNN))
%     \item …
% \end{itemize}
Stínění oken v pozemních stavbách určených k bydlení se využívá k regulaci teploty a zajištění soukromí obyvatel. (\cite{lubinova:stineni}) Jednou z možností stínění jsou venkovní žaluzie osazené pohonem na dálkové ovládání. Jejich uživatel může měnit výšku vytažení žaluzie a sklon lamel a ovlivňovat tím množství záření, které skrz okno prochází. Pomocí vzdáleného ovládání pak lze žaluzie řídit automaticky.

Komplexní systém může generovat akční zásahy pro řízení žaluzií na základě různých vstupních informací (datum a čas, měření veličin pomocí senzorů) bez účasti uživatele za účelem zajištění jeho komfortu a úspory prostředků vynaložených na regulaci teploty v interiéru (vytápění a chlazení). Sběr informací na základě kterých se má rozhodovat o nastavení žaluzie i samotné rozhodování musí probíhat v reálném čase.

Tato práce se zabývá návrhem takového systému, návrhem měřicích zařízení použitelných pro měření veličin užitečných při rozhodování systému a možnostmi využití strojového učení k ovládání žaluzií v reálném čase. Zkoumá časový vývoj použitých algoritmů při opakovaném učení na postupně sbíraných datech s ohledem na měnící se požadavky uživatele a měnící se podmínky v průběhu roku. Její cíle jsou následující:
\begin{enumerate}
    \item Určit vhodné vstupní veličiny (příznaky) a výstupní veličiny pro automatické rozhodování o stavu žaluzie.
    \item Zkonstruovat měřicí zařízení pro měřitelné příznaky odesílat jejich hodnoty pomocí MQTT\footnote{MQTT je komunikační protokol blíže popsaný v \refskl{sec:MQTT}{sekci}}.
    \item Zajistit komunikaci s vybranými žaluziemi pomocí poskytovaného API\footnote{\acrlong{api}} a MQTT\footnotemark[1].
    \item Zajistit komunikaci se zdroji všech příznaků pomocí MQTT a data ukládat pro pozdější použití.
    \item Navrhnout základní rozhodovací systém založený na pravidlech (baseline).
    \item Navrhnout model neuronové sítě a porovnat jeho použitelnost s baseline systémem.
\end{enumerate}

V práci se využívá skutečná venkovní žaluzie s hliníkovými lamelami a pohonem od firmy Somfy. Vzdálenou komunikaci s ním zajišťuje řídicí jednotka Tahoma připojená do internetu, která komunikuje se servery výrobce, jejichž prostřednictvím je možné žaluzii ovládat a zjistit její aktuální stav. Žaluzie se nachází před oknem pokoje autora v rodinném domě v obci vzdálené asi 12~km severovýchodně od Plzně. Uvnitř tohoto pokoje a na střeše daného domu také probíhají všechna měření.