\chapter{Discussion} \label{chap:discussion}

The overall objective of the thesis consisted of five subtasks:

\begin{enumerate}
    \item to test selected modules for use in the project;
    \item to implement hardware-dependent modules physically;
    \item to design an interface for communication between the user and selected modules;
    \item to add an interface for voice interaction;
    \item to enrich the project with other modules according to the time possibilities.
\end{enumerate}





% "The results indicate that..."

% ***Step 1***
% Give your interpretations
%     What do the results mean?
%     Identify correlations, patterns, and relationships among the data
%     Discuss whether the results met your excepctations or supported your hypotheses
%     "In line with the hypothesis..."

% ***Step 2***
% Discuss the implications
%     What has your research  contributed?
%     Do your results agree with previous research?
%     Are your findings very different from other studies?
%     Do the results support or challenge existing theories?
%     Are there any practical implications?
%     "The experiment provides a new insight into the relationship between..."

% ***Step 3***
% Acknowledge the limitations
%     "It is beyond the scope of this study to address the question of..."
   
% ***Step 4***
% State your recommendations (for further research and practical implementation)
%     For future research can lead directly from the limitations
%     Give concrete ideas for how future work can build on your research
%     "Further research is required to establish whether X is a factor in ..."

% CHECKLIST:
% I have concisely summarized the most important findings.
% I have discussed and interpreted the results in relation to my research questions.
% I have cited relevant literature to show how my results fit in.
% I have clearly explained the significance of my results.
% If relevant, I have considered alternative explanations of the results.
% I have stated the practical and/or theoretical implications of my results.
% I have acknowledged and evaluated the limitations of my research.
% I have made relevant recommendations for further research or action.

The thesis output indicates that with the help of now widely available technologies, building a smart home at a low cost is possible. As used in this project, everyone can use development boards, cheap sensors and motors, and free web hosting with this open-source project to make a home more like a smart home. A general open-source engine has been built to handle any module that complies (following the primary standard for connection to the engine). The user can communicate with modules in Czech, unlike Siri and other assistants. Thus, we can say that the output met our expectations to create an open-source, modularly functional system with voice-enabled modules for the smart home.

The thesis provides new insight into the relationship between cheap developing technology and available speech synthesis and speech recognition. When creating a module, it does not matter the principle it works, it can be based purely on Internet data, but it can also be linked to hardware. The module developer has several tools at his disposal, such as a database, a web interface and a communication channel that he can use. The thesis output provides an example of a practical implication. 

It is beyond the scope of this study to address the question of security and the impossibility of leaking personal data. Due to the complexity of this work, there is undoubtedly much room for improvement, such as adding more modules. However, the thesis gives a basic idea and direction for the future, where it should go. 

Further research is required to establish whether MQTT is set correctly for communication security. Further future research is essential to create a more efficient database and data storage, both for search speed and to save space.